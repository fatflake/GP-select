\documentclass[12pt]{article}
\usepackage{amsmath}
\usepackage{times}
\usepackage{graphicx}
\usepackage{color}
\usepackage{multirow}
%
%\usepackage[authoryear]{natbib}
%\usepackage{apacite}
%\usepackage{cite}
\usepackage[authoryear]{natbib}
%
\usepackage{rotating}
\usepackage{bbm}
\usepackage{latexsym}
%\DeclareGraphicsExtensions{.eps,.png}

\usepackage{subfigure} 
\graphicspath{../figs/}

% For algorithms
\usepackage{algorithm}
%\usepackage[]{algorithm2e}
\usepackage{algorithmic}

% --------------- our stuff --

\usepackage{amssymb}
\usepackage{amsmath}
%\usepackage{bbm}
%\usepackage{dsfont}
\setcounter{tocdepth}{3}
\usepackage{graphicx}
\usepackage{color}
\usepackage{caption}
%\usepackage{subcaption}
\usepackage{natbib}
\usepackage{url}

% ---------- commands ----------
\newcommand{\x}{\ensuremath{\times}}
\newcommand{\disT}{\textstyle}
\newcommand{\disS}{\displaystyle}
\newcommand{\prob}[2]{p(#1 \, | \, #2)}  % nicely space p(1 | 2)
\newcommand{\Prime}{\,'}  % nicely space p(1 | 2)
\newcommand{\One}{I}
\renewcommand{\vec}[1]{{\mathbf{#1}}}
\newcommand{\Kn}{\mathcal{K}_{n}}
\newcommand{\Ih}{\mathcal{I}_{n}}
\newcommand{\Sh}{\mathcal{S}_{h}}
\newcommand{\Ss}{\mathcal{S}}

\usepackage{float}

%\definecolor{Green}{rgb}{0,.80,0} 
% FIXME
%\newcommand{\comment}[1]{}
%\newcommand{\comment}[1]{\textcolor{Green}{[#1]}} % --- comment out when DONE with revisions ---
%\newcommend{\rev}[1]{\textcolor{Blue}{[#1]}}

% ---------- end our stuff ----------------

% reference info

% As of 2011, we use the hyperref package to produce hyperlinks in the
% resulting PDF.  If this breaks your system, please commend out the
% following usepackage line and replace \usepackage{icml2015} with
% \usepackage[nohyperref]{icml2015} above.
%\usepackage{hyperref}

% Packages hyperref and algorithmic misbehave sometimes.  We can fix
% this with the following command.
\newcommand{\theHalgorithm}{\arabic{algorithm}}


%%% margins 
\textheight 23.4cm
\textwidth 14.65cm
\oddsidemargin 0.375in
\evensidemargin 0.375in
\topmargin  -0.55in
%
\renewcommand{\baselinestretch}{2}
%
\interfootnotelinepenalty=10000
%
\renewcommand{\thesubsubsection}{\arabic{section}.\arabic{subsubsection}}
\newcommand{\myparagraph}[1]{\ \\{\em #1}.\ \ }
\newcommand{\citealtt}[1]{\citeauthor{#1},\citeyear{#1}}
\newcommand{\myycite}[1]{\citep{#1}}

% Different font in captions
\newcommand{\captionfonts}{\normalsize}

\makeatletter  
\long\def\@makecaption#1#2{%
  \vskip\abovecaptionskip
  \sbox\@tempboxa{{\captionfonts #1: #2}}%
  \ifdim \wd\@tempboxa >\hsize
    {\captionfonts #1: #2\par}
  \else
    \hbox to\hsize{\hfil\box\@tempboxa\hfil}%
  \fi
  \vskip\belowcaptionskip}
\makeatother   
%%%%%

\renewcommand{\thefootnote}{\normalsize \arabic{footnote}} 	

\begin{document}
\hspace{13.9cm}1

\ \vspace{20mm}\\ 

{\LARGE GP-select: Accelerating EM using adaptive \\subspace preselection}

\ \\
{\bf \large Jacquelyn A. Shelton$^{\displaystyle 1}$, Jan Gasthaus$^{\displaystyle 2,3}$, Zhenwen Dai$^{\displaystyle 4}$, J\"org L\"{u}cke$^{\displaystyle 5}$, and Arthur Gretton$^{\displaystyle 2}$}\\
{$^{\displaystyle 1 \,}$Technical University Berlin, Marchstrasse 23, 10587 Berlin, Germany.}\\
{$^{\displaystyle 2 \,}$University College London, Gatsby Unit, 25 Howland Street, London W1T 4JG, UK.}\\
%{$^{\displaystyle 2}$University College London, Gatsby Computational Neuroscience Unit, Sainsbury Wellcome Centre, 25 Howland Street, London W1T 4JG, UK.}\\
{$^{\displaystyle 3 \,}$Amazon Development Center, Karl-Liebknecht-Str. 5, 10178 Berlin, Germany.}\\
%{$^{\displaystyle 3}$Amazon Development Center Germany GmbH, Karl-Liebknecht-Str. 5, 10178 Berlin, Germany.}\\
{$^{\displaystyle 4 \,}$University of Sheffield, Western Bank, Sheffield, South Yorkshire S10 2TN, UK.}\\
{$^{\displaystyle 5 \,}$University of Oldenburg, Ammerl\"ander Heerstr. 114, 26129 Oldenburg, Germany.}\\
%
%Berlin, Germany \\
%\texttt{shelton@tu-berlin.edu} \\
%Berlin, Germany \\
%\texttt{j.gasthaus@ucl.ac.uk} \\
%\texttt{z.dai@sheffield.ac.uk} \\
%\texttt{joerg.luecke@uni-oldenburg.de} \\
%\texttt{arthur.gretton@gmail.com}
%\ \\[-2mm]
{\bf Keywords:} Approximate inference, generative graphical models, latent variable models, Expectation Maximization, EM acceleration, variable preselection

\thispagestyle{empty}
\markboth{}{NC instructions}
%
\ \vspace{-0mm}\\
%
%Abstract
\begin{center} 
{\bf Abstract} 
\end{center}
We propose a nonparametric procedure to achieve fast inference in generative graphical models when the number of latent states is very large.
 The approach is based on iterative latent variable preselection, where we alternate between learning a 'selection function' to reveal the relevant latent variables, and using this to obtain a compact approximation of the posterior distribution for EM; this can make inference possible where the number of possible latent states is e.g. exponential in the number of latent variables, whereas an exact approach would be computationally infeasible.
We learn the selection function entirely from the observed data and current EM state via Gaussian process regression. This is by contrast with earlier approaches, where selection functions were manually-designed for each problem setting.
We show that our approach performs as well as these bespoke selection functions on a wide variety of inference problems: in particular, for the challenging case of a hierarchical model for object localization with occlusion, we achieve results that match a customized state-of-the-art selection method,  at a far lower computational cost.


\input{gp_text_NCrevised}


%on a recent translation-invariant model, showing the ability of our approach to
%successfully do inference in a sophisticated hierarchical model -- with
%performance matching that of the model's complex hand-engineered selection
%function -- and while being applicable to larger scale problems.
%In these experiments, Gibbs sampling from the reduced latent space lead to further decrease computational costs.
%Furthermore, we show that with this approach we can do inference in challenging models and scale to a large number of latent variables.


\iffalse
\section{Citations}
The citations must follow the APA format. A typical citation is given by $\backslash citep$, for example \citep{Ref2009}. The command $\backslash cite$ will generate \cite{Ref2009}. For references with multiple authors, the command $\backslash citet$ will generate \citet{Ref2008} and $\backslash citep$ will generate \citep{Ref2008}. To put texts in the reference, use command $\backslash citep[see, e.g.,][for instance]\{Ref2009\}$ which generates \citep[see,
e.g.,][for instance]{Ref2009}. To cite multiple references, use the command $\backslash citep\{ref1,ref2,ref3\}$, for example \citep[compare][]{Ref2008,Ref2009}.
For Neural Computation, the $\backslash citep$ is preferred.
\fi

\subsection*{Acknowledgments}
We acknowledge funding by the German Research Foundation (DFG) under grant LU 1196/4-2 (JS), by the Cluster of Excellence EXC 1077/1 "Hearing4all" (JL), and by the RADIANT and WYSIWYD (EU FP7-ICT) projects (ZD).

%\section*{Appendix}




\bibliographystyle{apa}%cite} 
\bibliography{gp}

\iffalse
\begin{thebibliography}{100}
\providecommand{\natexlab}[1]{#1}
\expandafter\ifx\csname urlstyle\endcsname\relax
  \providecommand{\doi}[1]{doi:\discretionary{}{}{}#1}\else
  \providecommand{\doi}{doi:\discretionary{}{}{}\begingroup
  \urlstyle{rm}\Url}\fi

\bibitem[{LastName(2009)}]{Ref2009}
LastName, A. (2009).
\newblock Title for the first reference.
\newblock \emph{Journal of the first reference}, \emph{3}, 18 -- 88.


\bibitem[{Authors et~al.(2008)Author1, Author2, \&
  Author3}]{Ref2008}
Author1, A., Author2, A., \& Author3, A. (2008).
\newblock Title for the second reference.
\newblock \emph{Journal for the second reference}, \emph{5}, 188 -- 200.

\end{thebibliography}
\fi

\end{document}
